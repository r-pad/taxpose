% A figure with two images side-by-side
% \begin{figure}[ht]
% \centering
% \begin{subfigure}{.5\textwidth}
%   \centering
%   \includegraphics[scale=0.5]{images/fig1.png}
%   \caption{A subfigure}
%   \label{fig:sub1}
% \end{subfigure}%
% \begin{subfigure}{.5\textwidth}
%   \centering
%   \includegraphics[scale=0.5]{images/fig2.png}
%   \caption{A subfigure}
%   \label{fig:sub2}
% \end{subfigure}
% \caption{A figure with two subfigures}
% \label{fig:test}
% \end{figure}

% A figure with three images side-by-side
% \begin{figure}[ht]
% \centering
% \begin{subfigure}{.33\textwidth}
%   \centering
%   \includegraphics[width=\textwidth]{images/fig1.png}
%   \caption{A subfigure}
%   \label{fig:sub1}
% \end{subfigure}%
% \begin{subfigure}{.33\textwidth}
%   \centering
%   \includegraphics[width=\textwidth]{images/fig2.png}
%   \caption{A subfigure}
%   \label{fig:sub2}
% \end{subfigure}
% \begin{subfigure}{.33\textwidth}
%   \centering
%   \includegraphics[width=\textwidth]{images/fig3.png}
%   \caption{A subfigure}
%   \label{fig:sub3}
% \end{subfigure}
% \caption{A figure with three subfigures}
% \label{fig:test}
% \end{figure}
